%\chapter{Optimizing SNSPD HBT and TCSPC measurement of a fs pulsed source}
\chapter{Optimization of SNSPD-Based HBT and TCSPC measurements with a femtosecond pulsed source}

In this chapter will be presented the procedures carried out to optimize the HBT and TCSPC measurements that we took with a femtosecond pulsed laser source.
In detail, \autoref{cpp:Motivations} provides the reader the driving motivations as well as a description of the experimental setup we used.
Moving forward with the structure of the chapter, \autoref{cpp:Initial-khaos} provides conditions, observations and conclusions behind the first chaotic measurement obtained with this setting.
Lastly, \autoref{cpp:Improvements&discussion} brings to the discussion the driving idea behind the troubleshooting, together with a qualitative discussion of the results obtained after such changes.



\section{Motivations and experimental setup overview}
\label{cpp:Motivations}
When studying pulsed laser light with the use of an SNSPD as a detector, it is good practice to question the results obtained with respect to the various jitter sources involved. Isolating such timing jitter sources can be a difficult task. For this reason a good approach can involve changing the pulsed laser light to another whose jitter is known, or also negligible under certain circumstances. By choosing this approach we substantially simplify the recognition of all the other jitter sources involved outside of the pulse generation jitter, providing an optimal setup to quantify the precision of the detection instrument. 
In this second part of this thesis we will then present the procedures of optimization that we carried out on HBT and TCSPC measurements obtained with a femtosecond pulsed laser, whose intrinsic jitter can easily be considered negligible when compared to the laser source of the first part.

In particular, the laser source we used is a mode-locked Titanium-Sapphire (Ti:Sa) laser, driven by a 6.34~W Sprout-G-15W laser manufactured by Lighthouse Photonics. Under these working conditions the Ti:Sa is capable of producing $\leq 100$~fs pulses. An important remark needs to be made about this temporal duration : although the manufacturer specifies that with a 5~W pump laser the output pulses can reach $\sim 50$~fs, these values correspond to nearly transform-limited pulses under optimal cavity alignment and dispersion compensation. In our case the output was not transform-limited, so we treated the Ti:Sa as a black box. The only verification we carried out on this black box relied on the real time spectrum of the laser. From the spectral measurement we were able to identify the operating wavelength, the spectral width and above all confirm the pulsed regime. In the case of a spectrum without a distinct peak, we would have concluded that the Ti:Sa was producing CW output instead of the desired pulsed regime.

The key differences with respect to the experimental setup presented in \autoref{sec:Overview} can be summarized as follows:
\begin{itemize}
    \item Ti:Sa femtosecond pulses replacing the previous EOM-driven source,
    \item Single-mode fiber (Thorlabs 780HP) instead of the polarization-maintaining fiber,
    \item TCSPC operated with a divided SYNC provided by a multifunctional divider box, yielding an effective frequency at the ID1000 of about 20~MHz.
\end{itemize}

An overview of the experimental setup is shown in \autoref{TisaSetup}. The sketch does not include the attenuation stage placed before the fiber coupler, which is a fundamental part of the setup. This stage was specifically required to reduce the power from the original 0.1~W at the Ti:Sa output to roughly one quarter of that value. This downsizing was possible thanks to the use of a prism and a beam splitter, together with two infringing barriers. All these elements were properly positioned to direct only the last quarter of power into the fiber coupler.

%An overview of the experimental setup can be found in \autoref{TisaSetup}, though the sketch misses the part of the setup designed to attenuate the power before the fiber coupler. Although not depicted in the image, this is a fundamental part of the setup, specifically required to lower the power from the original 0.1~W at the raw exit of the Ti:Sa to roughly a quarter of it.

\begin{figure}[hbtp]
\centering
\includegraphics[width=1\textwidth]{TiSa_Setup.jpg}
\caption{Visual representation of the experimental setup. The pulsed output of the Ti:Sa laser (left) propagates through $\approx 10$~m of single-mode fiber (SMF) before reaching a fiber directional coupler. The beam is then divided into two channels that are directed to separate inputs of the SNSPD. Detection events are collected by the ID1000 time-tagger. Also shown are the electronic cables linking the Ti:Sa electronic pulser to the ID1000 via a multifunctional divider box.}
\label{TisaSetup}
\end{figure}




\section{Undesired Side Peaks in HBT and TCSPC measurements}
\label{cpp:Initial-khaos}
Up to this point in the discussion, we can now start presenting the data from the measurements.  
The first measurement was taken with an exposure time of 10~s, under conditions of approximately 3~MCounts detected. Regardless of the divided SYNC signal sent to the ID1000 at $\sim 20$~MHz, the actual repetition rate of the pulsed laser was 80~MHz.  

Assuming output pulses of $\approx 100$~fs, as discussed in the previous section, the fiber dispersion broadens them to $\approx 23$~ps at the SNSPD input. The resulting histograms are shown in \autoref{Khaos}, where each main peak is shifted to the center of the time axis, and the counts are displayed on a logarithmic y-axis.  

From these histograms, two key observations can be made:  
\begin{itemize}
	\item Although very similar overall, Detector~3 registers slightly more events than Detector~2, as indicated by the red main peak being $18.6\%$ higher than the black one;
	\item Side peaks appear at different relative times for the two detectors, but the $-500$~ps peak is consistently present across all four histograms.
\end{itemize}

From this first inspection, the conclusion is clear: before proceeding with further measurements, it is necessary to better characterize the origin of the side peaks observed here.


%with a non negligible number of side peaks appearing, it was necessary to take a step back to try assess the


\begin{comment}
At a regime of 3~MHz we ran the first set of measurements, over a time of 15s of exposure time.
The resulting histograms are then post-processed to shift the main peak in the zero time position, in order to ease the comparison between the different peaks.
\autoref{Khaos} shows the chaotic situation we got with the first measurement. So many side peaks appear, and the situation is far to be understood.

\end{comment}

\begin{figure}[hbtp]
\centering
\includegraphics[width=1\textwidth]{Khaos.jpg}
\caption{Measured TCSPC and HBT Histograms, post processed to display the main peak in the origin of the time axis to ease confrontation. Counts are displayed in the y-axis using a logarithmic scale. The two TCSPC histograms related to Detector~2 and Detector~3 are represented in black and red respectively. The two HBT histograms are displayed in green and blue.}
\label{Khaos}
\end{figure}

\section{Partly retrieving side peaks in HBT from coordinates of TCSPC peaks}
\label{cpp:Reconstructsides}
Explaining side peaks of a set of TCSPC and HBT histograms can be a difficult task. 
For this reason, a scenario like this one has to be treated carefully, starting from the basic informations that we know from the histograms.
According to what has been presented in \autoref{sec:Def-Techniques} the histograms arising from the two start/stop HBT  are strictly related to what we see in the two TCSPC plots related to just one detector.
For this reason, if we focus in \autoref{Khaos} to the total number of side peaks, we can notice how the two HBT are more affected by side peaks since they directly amplify the side peaks of the individual detectors that we see in the red and black plots. % NEEDS REFINEMENT, doesn't sound good, maybe it's better to have two separate sentences !!!

What we observe in \autoref{Khaos} can be explained mathematically by Eqs.~ \ref{TCSPC_convo} and \ref{HBT_convo} as follows :
\begin{equation}
TCSPC_{Det.2}(t) = I(t) \circledast IRF_{Det.2}(t)
\label{TCSPC_convo}
\end{equation}

\begin{equation}
HBT_{2,3}(t) = I(t) \circledast IRF_{Det.2}(t) \circledast IRF_{Det.3}(t)
\label{HBT_convo}
\end{equation}

The twofold convolution in \autoref{HBT_convo} makes it possible to qualitatively identify the HBT side peaks from the coordinates of the summits observed in the two separate TCSPC histograms.
Although qualitative, this procedure requires the time coordinates of every peak with reasonable accuracy.
A natural starting point is therefore to label each peak; from now on we adopt the labeling convention introduced in \autoref{Khaos_labeled}.
The corresponding numerical values are summarized in \autoref{tabellonadeipicchi}.

With this framework in place, the next step is to compute sums and differences of the C- and D-class peak coordinates, collecting the valid combinations in a list.
Among the several side peaks observed, four HBT peaks can be successfully identified through this procedure:
\begin{itemize}
\item $D0 - C1$ corresponds to the position of $A2$
\item $C1 - D0$ corresponds to the position of $B2$
\item $D1 - C0$ corresponds to the position of $A1$
\item $C0 - D1$ corresponds to the position of $B1$
\item The absence of $C4$ accounts for the reduced height of $A3$ relative to $B4$
\end{itemize}
These four identifications also serve as a direct confirmation that, by inverting the operations, one retrieves the positions of the corresponding peaks in the time-reversed HBT histogram. The hierarchical convention chosen for the numbering of the labels reinforces this interpretation: in both of the first two operations the resulting index is the same (2), and the same scheme applies to the third and fourth operations.



%Due to the convolution operator, these equations allow us to relate the peaks of the HBT histograms to the relative time coordinates of the TCSPC peaks.
%The main reason behind the possibility of reaching an explanation of the HBT peaks, from the TCSPC ones lies in the convolution operator, appearing twice, that relates the two detectors in 
\begin{figure}[hbtp]
\centering
\includegraphics[width=1\textwidth]{Khaos_Labeled.jpg}
\caption{Labeled version of \autoref{Khaos}. Each peak is labeled using the same color as its corresponding histogram. Numbers are assigned to indicate hierarchical relations between side peaks of the same color. The central peak is not labeled, but within this scheme it would correspond to the index zero.}
\label{Khaos_labeled}
\end{figure}


\begin{table}[h!]
\centering
\caption{\textbf{Detected position of the peaks}}
\renewcommand{\arraystretch}{1.3}
\begin{tabular}{
>{\centering\arraybackslash}m{1.5cm} 
>{\centering\arraybackslash}m{1.5cm} 
>{\centering\arraybackslash}m{1.5cm} 
>{\centering\arraybackslash}m{1.5cm} 
>{\centering\arraybackslash}m{1.5cm}}
\rowcolor{blue!50}
\textcolor{white}{\small[\textbf{ps}]} & \textcolor{white}{\textbf{A}} & \textcolor{white}{\textbf{B}} & \textcolor{white}{\textbf{C}} & \textcolor{white}{\textbf{D}} \\
\rowcolor{white}
\cellcolor{blue!50} \textcolor{white}{\textbf{0}} & 0    & 0    & 0     & 0     \\
\rowcolor{white}
\cellcolor{blue!50} \textcolor{white}{\textbf{1}} & 174  & -174 & 316   & 175   \\
\rowcolor{white}
\cellcolor{blue!50} \textcolor{white}{\textbf{2}} & -313 & 312  & -250  & -246  \\
\rowcolor{white}
\cellcolor{blue!50} \textcolor{white}{\textbf{3}} & 495  & -494 & -500  & +501  \\
\rowcolor{white}
\cellcolor{blue!50} \textcolor{white}{\textbf{4}} & -502 & 503  & n.d.  & -500  \\
\rowcolor{white}
\end{tabular}
\label{tabellonadeipicchi}
\end{table}









\section{Measurements with improved detection thresholds}
\label{cpp:Improvements&discussion}
This section is dedicated to the refinement of a cleaner dataset, with the ultimate goal represented by dataset possibly free of undesired side peaks.
Achieving such result would allow further analyses to be made, starting from the deconvolution of the IRF, for example.
Among the different approaches we tried, we will be presenting here only the most effective one, with a final discussion at the end.


\subsection{Driving idea}
Above other possible explanations and considering the small time intervals between the central peak and the other side summits, we concluded that the first part of the setup had to be the electronic one. We base this suspicion on behalf of the deadtime of the detector : the manufacturer states that  the recovery time before the nanowire is ready to detect an event after the previous one is far greater than the relative intervals we see in \autoref{Khaos}. It is then highly difficult that the physical reason behind those side peaks is some sort of afterpulsing. 
More likely is instead a scenario in which a poor choice of electronic thresholds in the ID1000 settings leads to the collection of more than one click per event detection in the SNSPD.
These hypotheses led us to the analysis of the electronic pulses generated by the SNSPD, to understand whether the settings were correct or not.
Up until this point, all measurements we took had the ID1000 set on triggering on the falling edge of the pulse, every time the electronic signal would dip below -100mV.

To analyze the electronic signal, we plugged the output cable of the second detector into an oscilloscope, making sure to add an impedance of 50~$\Omega$ to match the signal that was effectively reaching the time tagger.
Looking at \autoref{TrickShot} 



\begin{comment}
we strongly suspect that the chaotic behavior that we're getting, although we're visualizing the data always in log scale finds its origin in a sort of miscalibration of the triggering thresholds.
For this reason we managed to shut the laser and while the detector was receiving only stray photons, we looked at the electronic pulses coming out of the SNSPD.
Subsequently we initialized the Oscilloscope to visualize also the first derivative of the pulse so that we could get an info on where to locate the new, and refined threshold. The whole idea was to choose the point where the electrical signal expressed the greatest steepness, in order to achieve better measurements.
The whole idea we formulated explained the side peaks, as a consequence of having the detector measuring clicks when it was not fully recovered from the previous detection. By Modifying for a stronger threshold, as seen in \autoref{TrickShot}, we are getting rid of the even slow possibility to trigger with a non fully recovered nanowire.
Important mention is related to how we looked at these signals : instead of plugging the cable from the SNSPD straight into the Oscilloscope, we managed to add an impedence load to it, valued $50 \Omega$, to simulate the impedence that it will encounter at the entrance of the ID1000 Time-Tagger.
\end{comment}


\begin{figure}[hbtp]
\centering
\includegraphics[width=1\textwidth]{ScopeShots.jpg}
\caption{Oscilloscope view of an electronic pulse generated from two different SNSPD detectors. Left image is related to the second detector, while the right one is instead related to the third. The yellow line represent the signal itself, while the blue line represents its first derivative. Both the images do show the past and the refined thresholds, respectively like orange and green lines.}
\label{TrickShot}
\end{figure}

\begin{comment}
As seen in \autoref{TrickShot}, by looking at the images we significantly modified the thresholds, passing from a value of $-100mV$ for both the detectors to $-500mV$ and $-600mV$, for detector 2 and detector 3 respectively.
\end{comment}


\subsection{Discussion of the new measurements}
\begin{figure}[hbtp]
\centering
\includegraphics[width=1\textwidth]{CheckpointBravo_vs_Khaos.jpg}
\caption{Ciaone}
\label{RefinedMeasurement}
\end{figure}


% \subsection{Estimation of the width of the Detector response function}