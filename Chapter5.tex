\chapter{Conclusions and outlook}
\label{Capitolo5}


This thesis was motivated by the goal of improving single-photon indistinguishability for quantum applications. Two main experimental studies were carried out: analyzing HBT correlations from an electro-optically modulated pulsed laser, and optimizing TCSPC and HBT measurements with a femtosecond laser source. Before the experiments, a theoretical chapter introduced the basic principles behind the work.

The first study looked at correlations from a pulsed laser generated by two cascaded EOMs. A new method was developed to identify the zero-time delay peak, which allowed a more reliable analysis of $g^{2}(\tau)$ peak widths. The results show a steady increase in the timing jitter of the FPGA producing the electrical pulse pattern. 
%This indicates that the FPGA clock stability has a clear effect on how the CW laser is modulated into a pulsed source.

The second study focused on optimizing TCSPC and HBT measurements for a femtosecond pulsed laser. Even at the early stages, undesired side peaks were observed. Based on the SNSPD manufacturer’s test reports, it was clear that these peaks were not caused by detector after pulsing. Adjusting the time-tagger thresholds helped reduce some of the noise, but the final results still include an unexplained peak at +500 ps along with other distortions.

IIn short, this thesis proposed a practical method to identify the zero-delay peak in HBT correlation measurements and provided insights for improving measurement stability and reducing noise. Future work could involve testing a different pulse generator to evaluate the impact of FPGA-induced jitter on the results, and further investigating the unexplained +500 ps peak, which remains an open question.

\begin{comment}
%QUESTA è LA MIA VERISONE CHE HO SCRITTO.
The study presented in this thesis was motivated by the broader goal of boosting single photon indistinguishability for quantum applications. 
Within this context, two main experimental studies were carried out: A study on HBT correlations from a Electro-Optically modulated pulsed laser and the optimization of TCSPC and HBT measurements of a femtosecond laser source.
The results obtained provided insights into the stability of the FPGA clock governing the two cascaded EOMs and practical indications to increase the SNR in measurements obtained with an ID1000 time-tagger.

The thesis developed upon an initial theoretical chapter providing key elements for the understanding of the main topics, as well as the functioning principles that regulate the experimental techniques implemented.

Subsequently a comprehensive description of the two experimental chapters followed, providing an in depth description of the procedures implemented in the two experimental phases of this work.

The first study presented focused onto the insights obtained from the HBT correlations of a pulsed laser source obtained with the implementation of two cascaded EOMs.
Before introducing the experimental setup and its key features, we assessed the details of the correlations and found a new method to identify the zero-time delay peak. This identification enabled further analysis to be made, particularly focusing on $g^{2}(\tau)$ peak widths.
The final results of this part suggest a steady decrease of the timing jitter of the FPGA generating the electrical pulse pattern that dominates the modulation of the CW laser into a pulsed source. These conclusions arise from a critical analysis of \autoref{DeltaFWHMvsPERIOD_General}, an image centered upon the variations of the FWHM of the $g^{2}(\tau)$ peaks, with respect to the central peak.

The second part of the experimental work portrayed in this thesis was instead focused onto the optimization of TCSPC and HBT measurements of a pulsed fs-scaled laser. 
Although the initial purpose was to understand the detector response function of the same detection instrument of the first part by testing it with a different light source, characterized by less timing jitter, it was clear from the very early stages that such study could not be carried out without an initial understanding of the presence of several undesired peaks.
The subsequent study focused onto searching the mathematical behind the arising of some of these undesired side peaks.
The test reports conducted by the manufacturer of the SNSPD, provided the key information that guided the following steps, since the distances between the side peaks seen in the plots are far smaller than the related dead times.
For this reason upon looking with the oscilloscope at the pulses generated by the detector, the electrical thresholds of the time tagger were adjusted to remove undesired noise.
In conclusion, the final outcomes of this second experimental part did not meet the initial expectations.
\autoref{RefinedMeasurement} shows the final TCSPC and HBT measurements, with the clear presence of one, still unsolved, separate peak at +500ps from the center, together with several distortions appearing.

Further developments of this work could implement changing the FPGA with another pulse generator, to assess whether the timing jitter portrayed by the FWHM variations degrades or also improves. Regarding the second part, instead, the focus should be oriented towards understanding the physics of the unknown second peak appearing at +500ps.





%\begin{comment}
Cosa è necessario dire in questa parte ?!?!?!?!?

Parte 1 : Descrizione sintetica di cosa è stato raggiunto in questo lavoro ( COUPLE OF SENTENCES PER PART OF THESIS )

Parte EOM :
- discovered a new method to identify the central peak in the HBT correlations of a pulsed light source. 
		How we've been able to deliver this ? analyzing every peak with an automated script, and making considerations towards the width of these 
			peaks!!
- Observed that the jitter of the FPGA gets worse over longer shifts

Parte femtoseconds : Optimizing TCSPC and HBT measurements of a fs laser source !

- analyzed a poor measurement to understand if the math can explain some of the undesired mess happening.
		- went back to the theory and retrieved the time coordinates of the peaks of the HBT histograms, using the coordinates of the peaks detecte
			in the two TCSPC measurements.
			
- understood that refined thresholds concur to the achievement of better measurements, even though it is not fully resolutive as a solution.


Parte 2 : 

PARTE 2 : MAIN RESULTS OBTAINED IN THIS WORK

Parte EOM :
- discovered a new method to identify the central peak in the HBT correlations of a pulsed light source. 
		How we've been able to deliver this ? analyzing every peak with an automated script, and making considerations towards the width of these 
			peaks!!
- Observed that the jitter of the FPGA gets worse over longer shifts

Parte femtoseconds : Optimizing TCSPC and HBT measurements of a fs laser source !

- analyzed a poor measurement to understand if the math can explain some of the undesired mess happening.
		- went back to the theory and retrieved the time coordinates of the peaks of the HBT histograms, using the coordinates of the peaks detecte
			in the two TCSPC measurements.
			
- understood that refined thresholds concur to the achievement of better measurements, even though it is not fully resolutive as a solution.


%Parte 3 : ARE THERE ANY RELEVANT LIMITATIONS ?
%
%- GVD BROADENING. fs pulses gets widely spread out upon dispertion in the fiber.

Parte 4 : further developments to this work
\end{comment}