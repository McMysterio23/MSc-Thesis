\chapter{Conclusions and outlook}
\label{Capitolo5}


This thesis work comprises the study of timing jitter obtained by analyzing HBT correlations over an EOM-driven pulsed laser source.

Throughout the thesis the detector used for the measurements have been a SNSPD, whose



\begin{comment}
Cosa è necessario dire in questa parte ?!?!?!?!?

Parte 1 : Descrizione sintetica di cosa è stato raggiunto in questo lavoro ( COUPLE OF SENTENCES PER PART OF THESIS )

Parte EOM :
- discovered a new method to identify the central peak in the HBT correlations of a pulsed light source. 
		How we've been able to deliver this ? analyzing every peak with an automated script, and making considerations towards the width of these 
			peaks!!
- Observed that the jitter of the FPGA gets worse over longer shifts

Parte femtoseconds : Optimizing TCSPC and HBT measurements of a fs laser source !

- analyzed a poor measurement to understand if the math can explain some of the undesired mess happening.
		- went back to the theory and retrieved the time coordinates of the peaks of the HBT histograms, using the coordinates of the peaks detecte
			in the two TCSPC measurements.
			
- understood that refined thresholds concur to the achievement of better measurements, even though it is not fully resolutive as a solution.


Parte 2 : 

PARTE 2 : MAIN RESULTS OBTAINED IN THIS WORK

Parte EOM :
- discovered a new method to identify the central peak in the HBT correlations of a pulsed light source. 
		How we've been able to deliver this ? analyzing every peak with an automated script, and making considerations towards the width of these 
			peaks!!
- Observed that the jitter of the FPGA gets worse over longer shifts

Parte femtoseconds : Optimizing TCSPC and HBT measurements of a fs laser source !

- analyzed a poor measurement to understand if the math can explain some of the undesired mess happening.
		- went back to the theory and retrieved the time coordinates of the peaks of the HBT histograms, using the coordinates of the peaks detecte
			in the two TCSPC measurements.
			
- understood that refined thresholds concur to the achievement of better measurements, even though it is not fully resolutive as a solution.


%Parte 3 : ARE THERE ANY RELEVANT LIMITATIONS ?
%
%- GVD BROADENING. fs pulses gets widely spread out upon dispertion in the fiber.

Parte 4 : further developments to this work
\end{comment}