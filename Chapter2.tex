\chapter{Theory}
To investigate the optical jitter affecting the pulses of a pulsed laser source, it is first essential to establish a clear and simple definition of what jitter is. Building upon this foundational understanding, we can then introduce the experimental techniques and data analysis strategies used throughout this thesis.
\autoref{Sec:Def-Jitter} is dedicated to this purpose, beginning with a basic definition derived from signal theory, and progressively contextualizing it within the experimental framework explored in the following chapters.
\autoref{sec:Def-Pulses} provides the reader with the essential background on optical pulses, including key temporal properties and their connection to the spectral domain.
To better understand how jitter arises in a laboratory setting, \autoref{sec:Def-Sources} introduces a schematic overview of typical jitter sources and a classification framework.
Finally, \autoref{sec:Def-Techniques} offers the theoretical foundation behind the main experimental techniques used in this work, serving as a conceptual bridge between the theory and the measurements discussed in later chapters.
\section{Introduction to optical jitter}
\label{Sec:Def-Jitter}

To better understand what jitter means in the context of optical pulses, we must first clarify its general definition.

Jitter was originally defined in signal theory as the deviation of the timing instants of a sequence of events from their ideal positions in time. It can affect not only signals that are meant to be rhythmically constant—such as the ticks of a clock—but also repetitive or quasi-periodic signals that may include asynchronous events [IEEE Standard \cite{General_IEEE}].

In this thesis, we study jitter in the context of optical pulses, which serve as carriers of timing information for our experimental instruments and detectors. Given this focus, the more specific notion of timing jitter is most relevant.
According to the IEEE, timing jitter is defined as the deviation of the actual timing instants of a waveform from their ideal temporal positions \cite{General_IEEE}.


\section{Temporal characteristics of optical pulses}

\label{sec:Def-Pulses}
\section{Origins of jitter in experimental systems}
\label{sec:Def-Sources}
\section{Measuring and interpreting optical jitter}
\label{sec:Def-Techniques}