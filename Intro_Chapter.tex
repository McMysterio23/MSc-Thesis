\chapter{Introduction}

\begin{comment}
Motivations and guide to the general structure of the thesis among the different chapters !

Motivazioni : Quello di cui voglio parlare è inerente al contesto di sempre maggiore interesse al giorno d'oggi delle tecnologie sviluppate sulla base dei principi della quantum optics. 
A questo punto l'idea sarebbe quella di fare una breve picture storica, volta alla trasmissione al lettore del concetto secondo cui tutto questo ambito di ricerca sia estrememente recente e sebbene le prime avvisaglie sono state nei primi anni '50 con Glauber e HBT, il grosso arriva solo dagli anni 70-90.
Il tutto conduce all'inizio del nuovo secolo quando con l'avvento degli SNSPD e la teorizzazione e creazione dei primi QD si è iniziato a parlare di "Modern Quantum Optics".

La conclusione di questa parte di motivazioni io la farei andando ad introdurre il concetto del perfezionare la qualità della luce a singolo fotone che siamo in grado di produrre.    
\end{comment}

Modern society is heading towards the massive use of quantum computing and quantum communication protocols for research purposes, as well as hopefully one day like a revolutionary mean of communication.
All these technologies, far from being fully implemented with the state-of-the-art instruments find their functioning principles in the Quantum Optics theory. 
Although a first introduction of a quantum description of optics can be dated back in the early 50s with the works of Glauber, most of the interest for the topic rose from the 80s. The advent of the new century, then, was followed by the discovery and introduction of Single Photon Detectors and Quantum Dots : two truly revolutionary discoveries that mark nowadays modern quantum optics.

Within this context of rapidly developing races towards quantum computing, improving the single photon indistinguishability is a widely recognized topic of interest.

In this thesis we will present the study of the timing jitter of picosecond scaled optical pulses.

An important theoretical overview is provided in \autoref{Capitolo2}, where some key topics like timing jitter and optical pulses are accompanied by the working principles of the main experimental procedures implemented among the project.
In particular, the reader will be presented beforehand a comprehensive treatment of timing jitter from its essentials to some more inner insights.
Subsequently, an introduction to optical pulses is provided, spending some focus also towards the Time Bandwidth Product (TBP) one of the key properties.
The Chapter ends with a comprehensive description of the working principles of the main experimental techniques implemented widely within the project, namely the Hanbury, Brown, Twiss experiment and the Time-Correlated Single Photon Counting.

This initial theoretical framework represent the foundational environment implemented in the two following chapters, presenting the two experimental sections of this project.

\autoref{Capitolo3} presents the reader the study of electro-optically generated picosecond pulses, with the use of a modern day superconducting-nanowire single photon detector (SNSPD).
The reader will be presented the experimental setup and its features, to move on subsequently with a precise description of the operations conducted on the raw data. Results will be then presented and discussed in the final part of the chapter.


\autoref{Capitolo4} provides a good description of the second part of the project, focused on optimizing TCSPC and HBT measurements of a femtosecond pulsed laser source.
Although the final outcomes are naturally different, the structure of this chapter keeps the scheme of the previous one, starting with the description of the experimental setup and citing the key differences with the previous one.
The reader will understand the initial issues encountered, as well as the analysis and the following troubleshooting.
A final discussion is finally provided to assess whether the results obtained are encouraging or if no relevant progress has been made.

Finally, a summary of the project and a concise overview further developments is presented in \autoref{Capitolo5}.









%The current state-of-the-art of quantum communication has its functioning principles upon entangled states of single photons.
%This is a completely new field, characterized by a very recent and still developing theoretical background. 
%Although Quantum optics finds its roots in the 50s with the first discoveries brought by Glauber and Hanbury, Brown and Twiss, the key interest rose only in the decennia 70-90s.
%Modern day quantum optics gained even more interest with the discovery of single photon light sources like quantum dots, as well as the introduction of ultra-fast single photon detectors.
%All these technologies were developed altogether after the beginning of the new century. % that came into reality only at the beginning of the new century
%
%In recent years the perspectives of a widely spread use of totally safe mean of communication based on quantum communication principles, is attracting like never the interest of public and private entities.
%made possible by the
%The history of this research field, and its race towards a complete revolution of the means of communication currently drives the 